\ssr{ЗАКЛЮЧЕНИЕ}
В результате работы были изучены принципы функционирования,
построения и особенности архитектуры суперскалярных конвейерных микропроцессоров.

Были выполнены следующие задачи:

\begin{enumerate}
	\item Изучена и скомпилирована программа на ассемблере по варианту;
	\item Изучена временную диаграмму стадий выборки и диспетчеризацию команды по заданному адресу по варианту;
	\item Изучена временную диаграмму стадий декодирования и планирования команды по заданному адресу по варианту;
	\item Изучена временную диаграмму стадии выполнения команды по заданному адресу по варианту;
	\item Составлены трассы исходной и оптимизированной программ, проведено их сравнение.
\end{enumerate}

В результате оптимизации, не удалось уменьшить количество требуемых тактов для решения задачи, так как исходная программа была написана без конфликтов доступа к ресурсам и в ней не было задержек. В оптимизированной программе было уменьшено число сбросов конвейера из-за неудачных предсказаний блока предсказателя путём замены поиска максимума условным переходом на логические операции, что однако увеличило число тактов. Тем не менее для другого массива оптимизированная программа может быть эффективнее исходной.

Все цели и задачи лабораторной работы выполнены.